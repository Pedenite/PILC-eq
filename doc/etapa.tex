\documentclass[12pt]{report}

\usepackage[brazil,american]{babel}
\usepackage[utf8]{inputenc}

\usepackage{url}
\usepackage{graphicx}
\usepackage{authblk}
\usepackage{hyperref}
\usepackage{lipsum}
\usepackage{xcolor}

% \pagecolor[rgb]{0.1,0.1,0.1}
% \color[rgb]{0.9,0.9,0.9}

\begin{titlepage}
    \title{\includegraphics[width=4cm]{logo.jpg} \\ 
    \large{Dep. Ciência da Computação -- Universidade de Brasília (UnB)\\
        CIC0207 - Projeto Interdisciplinar de Licenciatura em Computação} \\
        \vfill 
        \vfill
        \LARGE
    \textbf{Portal de Equações de Primeiro e Segundo Grau}}

    \author{Letícia Dias Soares Alves, 18/0022059\\
        Pedro Henrique de Brito Agnes, 18/0026305
    }
    
    \affil{\vfill
        \vfill
        \vfill
        Professora \\
        Dr.a Letícia Lopes Leite
    }

    \date{Brasília\\Novembro de 2020}
\end{titlepage}

\begin{document}
\maketitle

\selectlanguage{american}
\begin{abstract}
  The aforementioned subject of mathematics can cause great fear on middle and high school students, who can have great difficulty on learning it. Our initial idea is to show examples of how to solve exercises step by step, examples of applications on other subjects and real world and an interactive mode to guide the user, all in a single website.
\end{abstract}

\selectlanguage{brazil}
\begin{abstract}
  O conteúdo de matemática citado acima causa um grande medo em alunos do ensino fundamental e médio, que apresentam grandes dificuldades no aprendizado do mesmo. Nossa proposta inicial é mostrar exemplos de resolução de exercícios passo a passo, aplicações em outras disciplinas e no mundo real e um modo interativo para auxiliar na aprendizagem, tudo por meio de um portal.
\end{abstract}

\tableofcontents
\newpage

\section{Introdução}
Durante o processo de desenvolvimento no raciocínio matemático, podem existir diversas falhas, tanto do lado do professor quanto do aluno, ou até mesmo do material didático. Por isso é de extrema importância o surgimento de novas ferramentas para auxiliar tanto o professor quanto os alunos.

Nosso trabalho será focado para as turmas do início do ensino médio, na disciplina de matemática, mais especificamente na matéria de equações (primeiro e segundo grau), pois essa é a matéria que muitos alunos apresentam dificuldades na hora de aprender e é importantíssima para outras matérias como física e química. Nosso software vai mostrar passo a passo de como resolver alguns exercícios dessas equações, além de mostrar os gráficos para o aluno assimilar mais facilmente a matéria.

\section{Detalhamento}
A dificuldade na resolução de equações ao longo do período escolar é natural e se torna presente em muitos alunos, que acabam por perder todo o interesse pela matéria, em especial quando não conseguem encontrar uma aplicabilidade para elas. A baixa capacidade de resolução de equações pode afetar negativamente o desempenho em matérias como física, química e a própria matemática, em que este conteúdo é usado o tempo todo e qualquer pequeno erro já faz com que o aluno perca toda uma questão. O projeto tem como finalidade, a descomplicação do conteúdo por meio de um \textit{website}.

Serão usadas as ferramentas HTML, CSS e Javascript para o desenvolvimento do portal que será composto apenas de \textit{front-end} e será versionado no \href{https://github.com/Pedenite/PILC-eq}{github}, onde também será feito o \textit{deploy} por meio do \href{https://pedenite.github.io/PILC-eq/}{github-pages}. Dentre o conteúdo do portal, teremos explicações da matéria considerando o público-alvo, que são jovens de 14 a 18 anos, exemplos de exercícios que vão variar desde os mais simples aos mais complexos e aplicações para proporcionar um entendimento melhor da matéria ao incluir gráficos e interdisciplinarizar com a física por exemplo e até coisas do mundo real.

\section{Trabalhos Relacionados e Conclusão}
Existem softwares na internet com uma proposta bem similar, como exemplo temos o \href{https://pt.symbolab.com/}{Symbolab} que permite que o usuário digite equações para que seja mostrada a solução detalhada passo a passo para o problema. Porém a ferramenta possui limitação no plano gratuito e não mostra todos os passos para a resolução, mas ela atende problemas mais avançados também como matérias estudadas no ensino superior.

Outro exemplo de um trabalho é o photomath. Trata-se de um aplicativo para smartphones que permite ao usuário apontar a câmera do celular em uma equação matemática e ele assume a responsabilidade de resolver mostrando gráfico, passos para resolução, entre outros. O aplicativo, além de resolver os problemas básicos de equações, também resolve funções mais complexas assim como o site citado acima.

A aplicação de interdisciplinaridade nos conteúdos de matemática já é usada há tempos em universidades em matérias como cálculo de forma a facilitar o entendimento e, principalmente fazer os alunos entenderem o "para quê" do conteúdo e não somente como resolver o problema, o que pode proporcionar um maior interesse pela matéria e fazer com que o conteúdo seja melhor fixado pelos alunos. Nosso objetivo é parecido neste sentido com a adição de ser um portal facilmente acessível pela internet para que qualquer um possa aprender.

\end{document}
